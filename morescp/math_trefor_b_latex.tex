\documentclass{article}
\usepackage[utf8]{inputenc}

\title{Hello World!}
\author{John M}
\date{Nov 30 2024}

\begin{document}

\maketitle

% *=takes away the number
	\section*{Introduction}

%enumerate=numberedList, itemize=bullets
\begin{enumerate}
\item Let's begin with a formula $e^{i\pi}+1=0$.

%pay attention to way command is typed and args it takes.
% \frac{}{}, \math_func, sqrt[]{}, etc

\item A more complicated formula
$$ e= \lim_{n\to\infty} \left(1+\frac{1}{n}\right)^n =
\lim_{n\to\infty}\frac{n}{sqrt[n]{n!}}$$

\item Polynomial Runtimes
	The are favorable r.t's ; 
$$ o(n), o(log n)$$

\item Ineffiecient / Bad Runtimes 
	(Opposite Poly)
	$$ O(n^2)$$

\item Linear Time
$$ O(n)$$

\item Constant Time
$$ O(1)$$

\item Logarithmic, Quasi(sub)-Logarithmic Time
$$ O(log_n)$$

\item Another formula
$$e=\sum_{n=0}^{\infty} \frac{1}{n!}.$$

\item We can also use continued fractions
$$e=2+\frac{1}{1+\frac{1}{2+\frac{2}{3+\ddots}}}$$

		\section*{More Math and Code Tutorial}
$$\begin{bmatrix}
	1 & 2&3\\
	4 &5&6\\
\end{bmatrix}
$$

\end{enumerate}

\end{document}



